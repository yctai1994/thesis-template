\begin{cjkAbstract}
    ある日の暮方の事である。一人の下人が、羅生門の下で雨やみを待っていた。
    広い門の下には、この男のほかに誰もいない。ただ所々丹塗りの剥げた、大きな円柱に、蟋蟀が一匹とまっている。羅生門が、朱雀大路にある以上は、この男のほかにも、雨やみをする市女笠や揉烏帽子が、もう二、三人はありそうなものだ。それが、この男のほかには誰もいない。

    なぜかというと、この二、三年、京都には地震とか辻風とか火事とか飢饉とかいう災いがつづいて起こった。そこで洛中のさびれ方は、一通りではない。旧記によると、仏像や仏具を打ち砕いて、その丹や金をぬすむ者が出たということである。洛中がその始末であるから、羅生門の修理などは、元より誰の気にもとまらなかった。

    するとその門の上には、そういう修理もせずに、捨てておかれた死骸が、いつとなく積み重ねられる事になった。しかも当時の感覚としては、この門の上で人間の死骸を見ても、別段驚くべき事とも感じなかった。――鴉が二三羽、何かくわえている。空模様も少し変って来た。

    下人は、そういう鴉の動作を見ながら、何とかしなければならぬと思っていた。
    しかし、何をどうしていいか、分らない。――旧記の記者の筆を借りれば、この男の考えは、羅生門の内外にある、他の幾千の考えと、何の違いもない。けれども、それでも、何とかしなければならぬと思っていた。
    \cjkAbstractKeywords
\end{cjkAbstract}
